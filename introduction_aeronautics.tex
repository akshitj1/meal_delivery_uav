\section{Introduction to Aeronautics}
Since, we assume that reader may not necessarily have Aeronautics/ Mechanical background, we will try to introduce to some basic concepts required to follow subsequent sections. For a detailed introduction, we recommend excellent texts\cite{anderson2005introduction}, \cite{anderson1999aircraft}.

Let us look at the simplest case of level unaccelerated flight. Drawing similarity from ground body.
TODO: insert figure
In order to balance all forces, we need to oppose gravity and oppose aerodynamic drag similar to friction in ground bodies. Force opposing gravity ie. weight($F_g$) is called lift($F_L$) and drag($F_D$) is compensated by generated thrust($F_T$).

Now, let us take case of earliest flying body we have seen ie. paper planes. To study aerodynamic forces a paper palne, we can assume flat plate theory\cite{tangler2005wind}.
\begin{align}
    \begin{split}
        \nonumber
        F_a &: \text{aerodynamic force} \\
        &= \frac{dp}{dt} = \frac{dm}{dt} v_n  \\
        &= v_n \frac{\rho A dx}{dt} = v_n (\rho A v_\infty) \\
        &= \rho A v_\infty^2 \sin(\theta) \\
    \end{split}\\
        \label{eq:flat_lift}
        F_L &= F_a \cos(\theta) = \rho A v_\infty^2 \sin(\theta) \cos(\theta)\\
        \label{eq:flat_drag}
        F_D &= F_a \sin(\theta) = \rho A v_\infty^2 \sin^2(\theta)\\        
    \begin{split}
        \nonumber
        \text{Here},\\
        v_\infty &: airspeed\\
        v_n&: \text{speed normal to surface}\\
        p&:\text{air momentum}\\
        A&:\text{area of plate}\\
        \theta&:\text{angle of attack}\\    
    \end{split}
\end{align}

\begin{align*}
%\label{eq:flat_drag}\\
\end{align*}

Note, even though this model makes too many oversimplifying assumptions(imcompressible, always seperated flow), it works surprisingly well to model\cite{cory2008experiments} and perch flat plate glider on wire\cite{moore2014robust}. The glider experiences angle of attack as high as 90\textdegree - 120\textdegree which are well beyond in stall regime(angle at which flow becomes seperated), yet data fits nicely with additional gaussian radial basis function.

Drag forces, are always produced as byproduct of lift as can be seen from \eqref{eq:flat_lift} and \eqref{eq:flat_drag}. This drag is called (lift)\textbf{induced drag}, while drag due to skin friction is called \textbf{parasitic drag}.Now that we know how to balance vetical forces ie. gravity $F_g$ with lift $F_L$, lets see how to generate thrust $F_T$ to oppose drag force $F_D$. Thrust is generated by propulsion sytems:
