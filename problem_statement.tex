\section{Problem Statement}
Assume, in an urban environment, a food order is placed by user to a restaurant by some means (App, call, or scheduled).
Once the order is received and prepared, it is dispatched from pickup location(restaurant) which will eventually reach 
customer. \\

Let,
\begin{align*}
    & t_{pl} : \text{time of order \textbf{pl}acement}\\
    & t_{pr}: \text{time order was \textbf{pr}epared}\\
    & t_{pk}: \text{time of order \textbf{p}ic\textbf{k}up}\\
    & t_{dr}: \text{time of order \textbf{dr}op(received by user)}\\
\end{align*}
Needless to say, 
\begin{align*}
    & t_j \in \mathbb{R^+} \forall t_j \in \{t_{pl}, t_{pr}, t_{pk}, t_{dr}\} \\
    \text{and} \\
    & 0 \leqslant t_{pl} < t_{pr} \leqslant t_{pk} < t_{dr}
\end{align*}

Using boldface $ \textbf{t}_x $ to represent time span between two events,
\begin{align*}
    \textbf{t}_{tot} & : \text{\textbf{tot}al time from order to drop} \\
    & = t_{dr} - t_{pl} \\
    \textbf{t}_{d} & : \text{time taken after it was ready for \textbf{d}elivery} \\
    & = t_{dr} - t_{pr}
\end{align*}

Usually there is a constraint,
\begin{equation}
    \textbf{t}_{tot}<\textbf{T}_{wait}  \label{wtime}
\end{equation}
where, $\textbf{T}_{wait}$ is the maximum time user can wait after placing an order.
eg. it won't be acceptable if your food arrives an hour after you order.
Less typically, there can also be constraint, 
\begin{equation}
    \textbf{t}_{d} < \textbf{T}_{stale}  \label{dtime}
\end{equation}
where $\textbf{T}_{stale}$ is the time after preparation, the item looses its integrity.
eg. a pizza looses its integrity after, say 30 mins if reheating is not availiable. If we had single order,
our goal would be to minimize the cost of delivery (we will define this cost later), while respecting constraints 
\eqref{wtime} and \eqref{dtime}. Note that, we can only control $\textbf{t}_{d}$ as $t_{pr}$ is within restaurant's control. 
So, if $t_{pr}-t_{pl}>\textbf{T}_{wait}$, then solution is trivially infeasible.

Concentrating on single delivery for the scope of this report, our aim will be to minimize cost $\boldsymbol{\theta_{\text{mission}}}$ incurred to take parcel from pickup point $x_p$ to drop point $x_d$ satisfying constraints put forth by vehicle $V$ and $\textbf{T}_{\text{wait}}$.
\begin{align*}
    & \displaystyle{\min_{\alpha \in S} \boldsymbol{\theta_{\text{mission}}}(\alpha)} \\
    & \alpha \doteq \{x_p, x_d, V\} \\
    & S \doteq S(V, T_{\text{wait}}) \\
\end{align*}
 In next section we will see, how $\boldsymbol{\theta_{\text{mission}}}$ or more generally $\boldsymbol{\theta_{a,b}}$ to move from point $x_a$ to $x_b$, depends on vehicle selection.


% If our vehicle takes path: \\
% \begin{equation*}
% p_{\text{mission}}=x_{\text{source}} \longrightarrow x_{\text{pickup}} \longrightarrow x_{\text{drop}}
% \end{equation*}
% and assuming cost of path theta_{source, drop} to be additive:

% where $p_{\text{mission}}$ is the path taken by vehicle to complete the misssion,
%  ${n_{\text{source}}}$ is the node corresponding to state of vehicle at its start location(could be a hub or charging station),
% $n_{\text{pickup}}$ is node corresponding to restaurant, $n_{\text{drop}}$ corresponds to user's location. Each $\longrightarrow$ is the edge $e_{a,b}$ for path from $n_a$ ot $n_b$.
% Cost of mission is:
% \begin{equation*}
%     \theta_{\text{mission}} = \Sigma\{ \theta_{a,b}\} \forall (a,b) \exists e_{a,b}  \in p_{\text{mission}}
% \end{equation*}
% In order to calculate $\theta_{mission}$, we need to find
% \begin{equation}
% \theta_{a,b}=f(n_a, n_b, s_{a}, V, t_{pk}, T_{tr})
% \end{equation}
% where,
% \begin{align*}
% n_x &: \text{state of node } x \vert x \in {a,b} \\
% s_a &: \text{state of vechicle at node } a \\
% V &: \text{vehicle constraints} \\
% \end{align*}
% We will now begin our journey to explore $f(\cdot)$ and how it varies with vehicle selection.